% Define document class
\documentclass[twocolumn]{aastex631}

% Filler text
\usepackage{blindtext}

% Begin!
\begin{document}

% Title
\title{An open source scientific article}

% Author list
\author{First Author}

% Abstract with filler text
\begin{abstract}
    \blindtext
\end{abstract}

% Main body
\section{Introduction}

Figure~\ref{fig:koch} shows an example of a figure with multiple subpanels.
The default behaviour for \texttt{showyourwork} is to assume that all the subpanels within a \texttt{figure} environment are generated by the same script.
In this case, by labeling the figure with \verb+\label{fig:koch}+, we are telling \texttt{showyourwork} that the script \texttt{figures/koch.py} generates the two PDF files included in the \texttt{figure} environment.

% A sample two-panel figure
\begin{figure}[ht!]
    \begin{centering}
        \includegraphics[width=0.4\linewidth]{figures/koch1.pdf}
        \includegraphics[width=0.4\linewidth]{figures/koch2.pdf}
        \caption{
            A sample plot with two subpanels, both generated from the same figure script.
            These show two iterations of a \href{https://en.wikipedia.org/wiki/Koch\_snowflake}{Koch snowflake}.
        }
        % This label tells showyourwork that the script `figures/koch.py'
        % generates the two PDF files included above
        \label{fig:koch}
    \end{centering}
\end{figure}

\end{document}
