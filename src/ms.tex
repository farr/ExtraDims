% Define document class
\documentclass[twocolumn]{aastex631}

% Filler text
\usepackage{blindtext}

% Begin!
\begin{document}

% Title
\title{An open source scientific article}

% Author list
\author{First Author}

% Abstract with filler text
\begin{abstract}
    \blindtext
\end{abstract}

% Main body
\section{Introduction}

Figure~\ref{fig:my_figure} is an example of a figure that depends on the output of an expensive computation.
In general, if a computation takes more than a few tens of minutes to run, you probably don't want to run it on GitHub Actions.
To help in these cases, \texttt{showyourwork} makes it easy to switch between running the computation and uploading the output to \href{https://zenodo.org}{Zenodo} (when running locally) and downloading the output from Zenodo (when running on GitHub Actions).
To set this up, simply specify the dataset(s) your figure script depends on in the top-level \texttt{showyourwork.yml} config file, along with a shell command to generate it and metadata for the Zenodo deposit.

% A sample figure generated from an external dataset
\begin{figure}[ht!]
    \begin{centering}
        \includegraphics[width=0.75\linewidth]{figures/my_figure.pdf}
        \caption{
            A sample figure generated from the output of a very expensive simulation.
            When running locally, the output is generated and uploaded to Zenodo;
            when running on the cloud, the output is downloaded from Zenodo.
            In addition to the usual GitHub icon in the margin of this caption, we also
            see an icon linking to the Zenodo record for the dataset, generated
            automatically by \texttt{showyourwork}.
        }
        % This label tells showyourwork that the script `figures/my_figure.py'
        % generates the PDF included above
        \label{fig:my_figure}
    \end{centering}
\end{figure}

\end{document}
