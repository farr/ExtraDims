% Define document class
\documentclass[modern]{aastex631}

\newcommand{\cca}{Center for Computational Astrophysics, Flatiron Institute, New York NY 10010, USA}
\newcommand{\sbu}{Department of Physics and Astronomy, Stony Brook University, Stony Brook NY 11794, USA}

\usepackage{blindtext}

% Begin!
\begin{document}

% Title
\title{Limits on the Number of Spactime Dimensions From GWTC-2 Observations}

% Author list
\author[0000-0002-9910-6782]{Kris Pardo}
\email{kpardo@caltech.edu}
\affil{Jet Propulsion Laboratory, California Institute of Technology, Pasadena, CA 91101, USA}

\author[0000-0003-1540-8562]{Will M. Farr}
\email{wfarr@flatironinstitute.org}
\email{will.farr@stonybrook.edu}
\affil{\cca}
\affil{\sbu}

% Abstract with filler text
\begin{abstract}
    \blindtext
\end{abstract}

\section{Introduction}
Gravitational waves provide us with important tests of general relativity and cosmology. LIGO etc.

GW170817 limits and tests of extra dims and gravity leakage. \citep{Pardo2018} (Pardo 2018, Lagos 2019, LIGO paper, Linder papers).

Although there have not yet been more GW events observed electromagnetically, the observed BBH event distribution can be used to place interesting cosmological bounds. Cite some other examples.

In particular, the mass function can be used to infer the redshifts of sources. Cite Farr et al 2019.

In this paper, we show how to use the PISN feature in the mass function to constrain gravitational leakage.

\section{Theoretical Framework}
Gravitational leakage basics (EM distance vs GW distance)

Discuss the different models: graviton decay, extra dims (with screening), and Horndeski $\alpha_M$.

Overview of PISN theory?

\subsection{Statistcal Formalism}

\section{Data Analysis}

\section{Results \& Discussion}

\bibliographystyle{aasjournal}
\bibliography{bib}

\end{document}
